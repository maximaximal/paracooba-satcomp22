\documentclass[conference]{IEEEtran}
\IEEEoverridecommandlockouts
% The preceding line is only needed to identify funding in the first footnote. If that is unneeded, please comment it out.
\usepackage{cite}
\usepackage{amsmath,amssymb,amsfonts}
\usepackage{algorithmic}
\usepackage{graphicx}
\usepackage{hyperref}
\usepackage{textcomp}
\usepackage{xcolor}
\def\BibTeX{{\rm B\kern-.05em{\sc i\kern-.025em b}\kern-.08em
    T\kern-.1667em\lower.7ex\hbox{E}\kern-.125emX}}

\newcommand{\paracooba}{\textsc{Paracooba}}
\newcommand{\kissat}{\textsc{Kissat}}
\newcommand{\cadical}{\textsc{CaDiCaL}}

\begin{document}

\title{Paracooba Enters SAT Competition 2022}

\author{\IEEEauthorblockN{Maximilian Levi Heisinger}
\IEEEauthorblockA{\textit{Institute for Symbolic Artificial Intelligence} \\
\textit{Johannes Kepler University Linz}\\
Linz, Austria \\
maximilian.heisinger@jku.at}
}

\maketitle

\paracooba{} is a parallel, distributed, Cube-and-Conquer (CnC) SAT solver based
on \kissat{}~\cite{kissatsat2020} and \cadical~\cite{cadical}. This year's
submission contains a few fixes compared to~\cite{satcomp2021}. \paracooba{}
tries to split problems into sub-problems by recursively setting variables to
concrete values, i.e. applying assumptions to the original formula. These
sub-problems are then solved with the incremental version of \cadical{},
combined with an (optional) timer to stop the solving process and re-split the
formula again. Next to this parallel solving process, an instance of \kissat{}
is also running sequentially to stop bad splits from having too large
impacts. The \paracooba{} solver can be found on GitHub using the URL below.

\begin{center}
\href{https://github.com/maximaximal/paracooba}{github.com/maximaximal/paracooba}
\end{center}

\section{Formula Splitting}

Splitting formulas is based on our implementation of tree-based lookahead~\cite{treelook} part of
\cadical{}. If the splitting with lookahead takes too long, we fall back to the number of
occurrences of variables. The chosen variable $x$ is then deemed to be the most decicive one and
used to split the formula $\phi$ into $\phi \wedge \neg x$ and $\phi \wedge x$. Now, if both
sub-formulas are unsatisfiable, the whole formula is also unsatisfiable. If one is satisfiable, the
whole formula is also satisfiable with the same assignment. This process is repeated until a
predefined cube-tree-depth is reached, which defaults to saturate the locally
available cores.
Sub-problems usually vary greatly in their hardness and are then solved in individual
solver threads or offloaded to other worker nodes in the network.

\section{Communication and Task Scheduling}

The communication between nodes uses a custom binary protocol transported via
TCP. After starting, a fully interconnected network of workers is created.
Every worker receives utilization information of all other worker nodes and is
thus able to decide locally, whether to offload work from the local queue to
other nodes in the network. The dynamic offloading of tasks ensures that the
highly different task hardness is mitigated by always saturating as many cores
as possible.

To start, paracooba does not need the addresses of it's workers. The main node
can start on it's own and either immediately start solving (if local worker
threads were enabled) or idle until worker nodes connect to it. The main node
listens on a port for incoming connections from workers (both the port and
listen address are configurable). After connecting, workers receive all other
known peers from the peer they connected to, in turn forming the fully connected
network explained above.  Workers may solve multiple problems at once, problems
sharing the available worker threads on a given node.

\section{Extending Paracooba}

\paracooba{} is built to be highly modular, also offering a QBF solving module.
The software is MIT-licensed and can be found on GitHub. More details about the
implementation, the employed algorithms and the expandability can be found in
\cite{paracooba} and \cite{heisinger2021distributed}. Possible extension points
are the communication stack, the solver module, the offloading / scheduling
mechanism, and the local task runner. Extensions may be loaded from provided
shared object files at start-up. Automated testing of modules is done using
integration tests, also provided in the repository.

\begin{thebibliography}{00}
\bibitem{kissatsat2020}{A. Biere, K. Frazekas, M. Fleury, M. Heisinger, CaDiCaL,
    Kissat, Paracooba, Plingeling and Treengeling entering the SAT competition
    2020. SAT Competition 2020}
\bibitem{cadical}{A. Biere, CaDiCaL at the SAT Race 2019, SAT Race 2019}
\bibitem{satcomp2021}{A. Biere, M. Fleury, M. Heisinger, CaDiCaL, Kissat,
    Paracooba Entering the SAT Competition 2021, SAT Competition 2021}
\bibitem{treelook}{M. Heule, M. Järvisalo, A. Biere, Revisiting hyper binary resolution,
International Conference on Integration of Constraint Programming, Artificial Intelligence, and
Operations Research. Springer, Berlin, Heidelberg, 2013}
\bibitem{paracooba}{M. Heisinger, M. Fleury, A. Biere, Distributed Cube and Conquer with Paracooba,
SAT2020}
\bibitem{heisinger2021distributed}{M. Heisinger, Distributed SAT \& QBF Solving: The Paracooba
Framework, 2021}
\end{thebibliography}

\end{document}
